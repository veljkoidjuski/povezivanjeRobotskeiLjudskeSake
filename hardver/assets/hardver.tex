\input amstex
\input epsf

\documentstyle{amsppt}

\magnification=\magstep1

\def\dj{d\kern-0.5em\raise0.3em\hbox{--}}

\topmatter

\title
Hardver projekta
\endtitle

\abstract
Ovaj model nazvan je ,,kinderica'' jer oblikom podse\'ca na \v{c}okoladice kinder. On opisuje kretanje prsta na\v{s}e robotske \v{s}ake pri povla\v{c}enju tetive. Potreban je da bi se iz koordinata vrha prsta dobijenih trijangulacijom odlu\v{c}ilo koliko je potrebno zarotirati odgovaraju\'ci servo motor.
\endabstract

\author
Sekcija robotike ST\v{S} Sombor
\endauthor

\endtopmatter

\head \v{S}aka \endhead

Glavna hardverska komponenta projekta jeste prototip robotske \v{s}ake, koji je sa\v{c}injen od \v{s}perplo\v{c}e koja je osnova \v{s}ake i komada ba\v{s}tenskog creva koji odgovaraju ljudskim prstima.

Svaki prst use\v{c}en je na takvim mestima da useci vr\v{s}e ulogu zglobova. Useci su tako napravljeni da verno opona\v{s}aju kretanje realne ljudske \v{s}ake.

Za upravljanje prstima napravili smo tetive, za \v{s}ta smo upotrebili tanke sinteti\v{c}ke kanape. Jedan kraj svakog kanapa pri\v{c}vr\v{s}\'cen je za unutra\v{s}nju stranu vrha svakog prsta.
\bye
