\input amstex
\input epsf
\input eplain

\documentstyle{amsppt}

\magnification=\magstep1

\beginpackages
\usepackage{graphicx}
\endpackages

\def\dj{d\kern-0.5em\raise0.3em\hbox{--}}

\topmatter

\title
Hardver projekta
\endtitle

\abstract
Ovaj model nazvan je ,,kinderica'' jer oblikom podse\'ca na \v{c}okoladice kinder. On opisuje kretanje prsta na\v{s}e robotske \v{s}ake pri povla\v{c}enju tetive. Potreban je da bi se iz koordinata vrha prsta dobijenih trijangulacijom odlu\v{c}ilo koliko je potrebno zarotirati odgovaraju\'ci servo motor.
\endabstract

\author
Sekcija robotike ST\v{S} Sombor
\endauthor

\endtopmatter


\specialhead SISTEM $\pmb\alpha$ \endspecialhead


Sistem $\alpha$ bavi se prihvatom i obradom podataka. \v{C}ine ga IBM-PC-kompatibilan ra\v{c}unar (trenutno Thinkpad X280) i dve kamere (Microsoft LifeCam VX-3000) postavljene na u horizontalnoj ravni, tako da su im sopstvene $x$ ose ortogonalne, a $y$ ose paralelne. Za potrebe fiksiranja polo\v{z}aja kamera izra\dj en je naro\v{c}it stalak od kartona. (Slika 1)

\specialhead SISTEM $\pmb\omega$ \endspecialhead

Sistem $\omega$ jeste izvr\v{s}ni element projekta. Njegova elektri\v{c}na \v{s}ema mo\v{z}e se videti na slici 2. \v{C}ine ga robotska \v{s}aka, servo motori i platforma Arduino Uno.

\head \v{S}aka \endhead

Glavna hardverska komponenta projekta jeste robotska \v{s}aka (slika 3), koji je sa\v{c}injen od \v{s}perplo\v{c}e koja je osnova \v{s}ake i komada ba\v{s}tenskog creva koji odgovaraju ljudskim prstima.

Svaki prst use\v{c}en je na takvim mestima da useci vr\v{s}e ulogu zglobova. Useci su tako napravljeni da verno opona\v{s}aju kretanje realne ljudske \v{s}ake.

Za upravljanje prstima napravili smo tetive, za \v{s}ta smo upotrebili tanke sinteti\v{c}ke kanape. Jedan kraj svakog kanapa pri\v{c}vr\v{s}\'cen je za unutra\v{s}nju stranu prsta, a drugi namotan na odgovaraju\'ci kotur. Ukupno ima dva kanapa po prstu, \v{s}to \v{c}ini da svaki prst ima dva stepena slobode. Trenutno se koristi samo jedna sajla po prstu (ona pri\v{c}vr\v{s}\'cena pri vrhu), tako da je u upotrebi 5 servo motora, a prsti imaju po jedan stepen slobode (slika 4). Druga sajla uti\v{c}e samo na zglob najbli\v{z}i dlanu i njeno potpuno izvla\v{c}enje, bez povla\v{c}enja prve tetive, \v{c}ini da se ceo prst postavi (gotovo) normalno na ravan dlana (slika 5).


\head Servo motori i njihova uloga \endhead

Veoma va\v{z}ni elektromehani\v{c}ki aktuatori u na\v{s}em projektu su pozicioni servo motori koji omogu\v{c}avaju regulisano kretanje prstiju.

Servo motori su elektromehani\v{c}ki ure\dj aji. Postoje oni koji precizno reguli\v{s}u polo\v{z}aj, tj. ugao otklona, oni koji precizno reguli\v{s}u brzinu obrtanja i oni koji precizno reguli\v{s}u obrtni moment.

Koristimo servo motore SG-5010 (slika 6), ugra\dj ene u podlakticu \v{s}ake. Na njihove rotore pri\v{c}vr\v{s}\'ceni su 3D \v{s}tampani koturovi na koje su namotane tetive.

Za upravljanje servo motorima koristi se mikrokontrolerska platforma Arduino Uno (slika 7), koji generi\v{s}e odgovaraju\'ce PWM signale za svaki motor. Motori se, zbog velike ukupne snage, napajaju laboratorijskim napajanjem.

\vfill

\pagebreak

\hbox to \hsize
{
\vbox to \vsize{
\phantom{m}
\vfill
\hsize=0.4\hsize
\centerline{\includegraphics[keepaspectratio, width = 0.9\hsize]{alfa.jpg}}
\vskip\baselineskip
\centerline{ Slika 1. \it Sistem $\alpha$}
\vskip\baselineskip
\vfill
\phantom{m}
}

\hskip-4em

\vbox to \vsize
{
\phantom{m}
\vfill
\centerline{\includegraphics[keepaspectratio, height = .6\vsize]{sema.png}}
\vskip\baselineskip
\centerline{ Slika 2. \it elektri\v{c}na \v{s}ema sistema $\omega$}
\vskip\baselineskip
\vfill
\phantom{m}
}
}



\pagebreak
\phantom{m}
\vfill

\centerline{\includegraphics[keepaspectratio, width = .5\hsize]{saka1.jpg}}
\vskip\baselineskip
\centerline{ Slika 3. \it robotska \v{s}aka}
\vskip\baselineskip

\vfill

\centerline{\includegraphics[keepaspectratio, width = .5\hsize]{saka2.jpg}}
\vskip\baselineskip
\centerline{ Slika 4. \it polo\v{z}aj prsta pri maksimalnom izvla\v{c}enju tetive koja je u upotrebi}
\vskip\baselineskip

\vfill

\pagebreak
\phantom{m}
\vfill

\centerline{\includegraphics[keepaspectratio, width = .5\hsize]{saka3.jpg}}
\vskip\baselineskip
\centerline{ Slika 5. \it polo\v{z}aj prsta pri maksimalnom izvla\v{c}enju druge tetive}
\vskip\baselineskip

\vfill

\hbox to \hsize
{
\vbox
{
\hsize=0.4\hsize

\centerline{\includegraphics[keepaspectratio, width = \hsize]{motor.png}}
\vskip\baselineskip
\centerline{ Slika 6. \it kori\v{s}\'ceni servo motor}
\vskip\baselineskip

}

\hfill

\vbox
{
\hsize=0.4\hsize

\centerline{\includegraphics[keepaspectratio, width = \hsize]{arduino.jpg}}
\vskip\baselineskip
\centerline{ Slika 7. \it platforma Arduino Uno}
\vskip\baselineskip


}


}

\vfill


\bye
